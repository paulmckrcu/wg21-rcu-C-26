%!TEX root = rcu.tex
%%--------------------------------------------------
%% Title page for C++ Technical Specification


\thispagestyle{empty}
\begingroup
\def\hd{\begin{tabular}{ll}
          \textbf{Document Number:} & {\larger\docno}             \\
          \textbf{Date:}            & \reldate                    \\
          \textbf{Revises:}         & \prevdocno                  \\
          \textbf{Reply to:}        & Paul E. McKenney            \\
                                    & Meta                        \\
                                    & paulmckrcu@gmail.com
          \end{tabular}
}
\newlength{\hdwidth}
\settowidth{\hdwidth}{\hd}
\hfill\begin{minipage}{\hdwidth}\hd\end{minipage}
\endgroup

\vspace{2.5cm}
\begin{center}
\textbf{\Huge
Why RCU Should be in C++26} \\
\vspace{1cm}
\emph{Authors:} \\
Paul McKenney, Michael Wong, Maged M. Michael, Geoffrey Romer, Andrew Hunter,
Daisy Hollman, JF Bastien, Hans Boehm, David Goldblatt, Frank Birbacher,
Erik Rigtorp, Tomasz Kamiński \\
\vspace{0.5cm}
\emph{email:} \\
paulmckrcu@fb.com, michael@codeplay.com, maged.michael@acm.org,
gromer@google.com, andrewhhunter@gmail.com, dhollman@google.com,
cxx@jfbastien.com, hboehm@google.com, davidtgoldblatt@gmail.com,
frank.birbacher@gmail.com, erik@rigtorp.se, tomaszkam@gmail.com
\end{center}
\vfill
% \textbf{Note: this is an early draft. It's known to be incomplet and
%   incorrekt, and it has lots of
%   b\kern-1.2pta\kern1ptd\hspace{1.5em}for\kern-3ptmat\kern0.6ptti\raise0.15ex\hbox{n}g.}
\newpage
