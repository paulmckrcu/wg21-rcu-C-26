%!TEX root = ts.tex

\chapter{Introduction}
\label{chp:Introduction}

We propose RCU for inclusion into C++26.
This paper contains proposed rationale to support RCU into C++26 as
well as the interface and wording for RCU, a technique for safe deferred
reclamation.

\section{Proposed Entry to C++26 IS}
\label{sec:Proposed Entry to C++26 IS}

A near-superset of this proposal is implemented in the Folly RCU library.
This library has used in production for several years, so we have good
implementation experience for the proposed variant of RCU.

This proposal is identical to that in Concurrency TS 2.
We expect that the proposal in Concurrency TS 2 will change over
time, for example, adding some of the features that are present in
the Folly RCU library or in the Linux kernel.
Such features might include:

\begin{enumerate}
\item	Multiple RCU domains.
	For example, SRCU provides these in the Linux kernel.
	However, RCU was in the Linux for four years before
	this was needed, so it is not yet in the TS nor is it
	in the proposal for C++26.
\item	Special-purpose RCU implementations.
	For example, the Linux kernel has specialized implementations
	for preemptible environments, single-CPU systems,
	as well as three additional implementations required by
	the Linux kernel's tracing and extended Berkeley Packet
	Filter (eBPF) use cases.
	However, none of these seem applicable to userspace applications,
	so none of them are yet in the TS or yet in in the proposal
	for C++26.
\item	Polling grace-period-wait APIs.
	These allow non-blocking algorithms to interface with RCU
	grace periods, for example, in the Linux kernel, they allow
	NMI handlers to do RCU updates.
	(NMI handlers could do RCU readers from the get-go.)
	However, RCU was in the Linux kernel for more than a decade
	before such APIs were needed, so they are not yet in the
	TS nor are they in the proposal for C++26.
\item	Numerous efficiency-oriented APIs.
	For but one example, the Linux kernel has an alternative
	\tcode{rcu_access_pointer()} that can be used in place
	of \tcode{rcu_dereference()} (Linux-kernelese for ``consume load'')
	when the resulting pointer will not be dereferenced
	(for example, when it is only going to be compared to \tcode{NULL}).
	But it is not clear which (if any) of these would be accepted
	into the Linux kernel today, given the properties of modern
	computer hardware.
	Therefore, these are not yet in the TS nor are they in the
	proposal for C++26.
\end{enumerate}

The snapshot library described in P0561R5 (``RAII Interface for Deferred
Reclamation'') provides an easy-to-use deferred-reclamation facility
applying only to a single object which is intended to be based upon
either RCU or Hazard Pointers.
It cannot replace either RCU or Hazard Pointers.

\begin{table*}
\renewcommand*{\arraystretch}{1.25}
\footnotesize
\centering
\begin{tabular}{l|lll}
	Property
		& Reference Counting
			& Hazard Pointers
				& RCU \\
%		  RC	  HP	  RCU \\
	\hline
	\hline
	Readers
		& Slow and unscalable
			& \emph{Fast and scalable}
				& \emph{Fast and scalable} \\
	\hline
	Unreclaimed Objects
		& \emph{Bounded}
			& \emph{Bounded}
				& Unbounded \\
%	\hline
%	Protection Duration
%		& Can be long
%			& Can be long
%				& Must bound duration \\
	\hline
	Traversal Retries?
		& If object deleted
			& If object deleted
				& \emph{Never} \\
	\hline
	Reclamation latency?
		& \emph{Fast}
			& Slow
				& Slow \\
\end{tabular}
\caption{High-Level Comparison of Deferred-Reclamation Techniques}
\label{tab:High-Level Comparison of Deferred-Reclamation Techniques}
\end{table*}

The Hazard Pointers library described in PNNNNR0 (``Hazard Pointers:
Proposed Interface and Wording for C++26'') is an alternative deferred
reclamation technique.
As a very rough rule of thumb, Hazard Pointers can be considered to be
a scalable replacement for reference counters and RCU can be considered
to be a scalable replacement for reader-writer locking.
A high-level comparison of reference counting, Hazard Pointers, and RCU
is displayed in
Table~\ref{tab:High-Level Comparison of Deferred-Reclamation Techniques}.

Note that we are making this working paper available before Concurrency
TS2 been published, which some might feel is unconventional.
On the other hand, Paul was asked to begin this effort in 2014, it is now
2022, and C++ implementations have been used in production for some time,
perhaps most notably the Folly RCU library.

\section{History}
\label{sec:History}

This paper is derived from N4895, which was in turn based on P1122R4.

P1122R4 is a successor to the RCU portion of P0566R5, in response to
LEWG’s Rapperswil 2018 request that the two techniques be split into
separate papers.

This is proposed wording for Read-Copy-Update [P0461], which is
a technique for safe deferred resource reclamation for optimistic
concurrency, useful for lock-free data structures.
Both RCU and hazard pointers have been progressing steadily through SG1
based on years of implementation by the authors, and are in wide use in
MongoDB (for Hazard Pointers), Facebook, and Linux OS (RCU).

We originally decided to do both papers' wording together to illustrate
their close relationship, and similar design structure, while hopefully
making it easier for the reader to review together for this first
presentation.
As noted above, they have been split on request.

This wording is based P0566r5, which in turn was based on that of on
n4618 draft [N4618].

\section{Source-Code Access}
\label{sec:Source-Code Access}

The Folly library is open source, and its RCU implementation may be
accessed here:

\begin{itemize}
\item	https://github.com/facebook/folly/blob/main/folly/synchronization/Rcu.h
\item	https://github.com/facebook/folly/blob/main/folly/synchronization/Rcu-inl.h
\item	https://github.com/facebook/folly/blob/main/folly/synchronization/Rcu.cpp
\end{itemize}

There is an additional reference implementation of this proposal.
Unlike the Folly library's version, this reference implementation is
not production quality.
However, it is quite a bit simpler, having delegated the difficult parts
to the C-language userspace RCU library:

\begin{itemize}
\item	https://github.com/paulmckrcu/RCUCPPbindings/tree/master/Test/paulmck
\item	https://liburcu.org
\end{itemize}


\section{Acknowledgments}
\label{sec:Acknowledgments}

We owe special thanks to Jens Maurer and Arthur O'Dwyer for their many
contributions to this effort.
