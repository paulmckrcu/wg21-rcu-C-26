%!TEX root = ts.tex

\chapter{Introduction}
\label{chp:Introduction}

This paper proposes addition of RCU to C++26.

\section{Proposed Entry to C++26 IS}
\label{sec:Proposed Entry to C++26 IS}

A near-superset of this proposal is implemented in the Folly RCU library.
This library has used in production for several years.

This proposal is identical to that in Concurrency TS 2.
We expect that the proposal in Concurrency TS 2 will change over
time, for example, adding some of the features that are present in
the Folly RCU library.
However, we have good implementation experience for the proposed
variant of RCU.

The snapshot library described in P0561R5 (``RAII Interface for Deferred
Reclamation'') provides an easy-to-use deferred-reclamation facility
applying only to a single object which is intended to be based upon
either RCU or Hazard Pointers.

The Hazard Pointers library described in PNNNNR0 (``Hazard Pointers:
Proposed Interface and Wording for C++26'') is an alternative deferred
reclamation technique.
As a very rough rule of thumb, Hazard Pointers can be considered to be
a scalable replacement for reference counters and RCU can be considered
to be a scalable replacement for reader-writer locking.

Note that we are making this working paper available before N4895 has
been published, which some might feel is unconventional.
On the other hand, Paul was asked to begin this effort in 2014, it is now
2022, and C++ implementations have been used in production for some time,
perhaps most notably the Folly RCU library.

\begin{table*}
\renewcommand*{\arraystretch}{1.25}
\footnotesize
\centering
\begin{tabular}{l|lll}
	Property
		& Reference Counting
			& Hazard Pointers
				& RCU \\
%		  RC	  HP	  RCU \\
	\hline
	\hline
	Readers
		& Slow and unscalable
			& Fast and scalable
				& Fast and scalable \\
	\hline
	\# Protected Objects
		& Scalable
			& Unscalable
				& Scalable \\
	\hline
	Protection Duration
		& Can be long
			& Can be long
				& Must bound duration \\
	\hline
	Traversal Retries?
		& If object deleted
			& If object deleted
				& Never \\
\end{tabular}
\caption{Comparison of Deferred-Reclamation Techniques}
\label{tab:Comparison of Deferred-Reclamation Techniques}
\end{table*}

\section{History}
\label{sec:History}

This paper is derived from N4895, which was in turn based on P1122R4.

P1122R4 is a successor to the RCU portion of P0566R5, in response to
LEWG’s Rapperswil 2018 request that the two techniques be split into
separate papers.

This is proposed wording for Read-Copy-Update [P0461], which is
a technique for safe deferred resource reclamation for optimistic
concurrency, useful for lock-free data structures.
Both RCU and hazard pointers have been progressing steadily through SG1
based on years of implementation by the authors, and are in wide use in
MongoDB (for Hazard Pointers), Facebook, and Linux OS (RCU).

We originally decided to do both papers' wording together to illustrate
their close relationship, and similar design structure, while hopefully
making it easier for the reader to review together for this first
presentation.
As noted above, they have been split on request.

This wording is based P0566r5, which in turn was based on that of on
n4618 draft [N4618].
